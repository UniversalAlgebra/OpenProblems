\begin{filecontents*}{citations.bib}
@article {mckenzie-residual-bounds,
     AUTHOR = {McKenzie, Ralph},
     TITLE = {The residual bounds of finite algebras},
     JOURNAL = {International Journal of Algebra and Computation},
    VOLUME = {6},
   YEAR = {1996},
   NUMBER = {1},
   PAGES = {1--28},
   ISSN = {0218-1967},
   note =  {doi: 10.1142/S0218196796000027},
   URL =  {http://dx.doi.org/10.1142/S0218196796000027},
}
@article{haddad,
  year={1990},
  issn={0002-5240},
  journal={Algebra Universalis},
  volume={27},
  number={2},
  doi={10.1007/BF01182451},
  title={Intersections of finitely generated clones},
  url={http://dx.doi.org/10.1007/BF01182451},
  publisher={Birkhäuser-Verlag},
  author={Haddad, Lucien},
  pages={171-179},
  language={English}
}
@ARTICLE{tardos,
   author={G\'abor Tard\H{o}s},
  title="A maximal clone of monotone operations which in nonfinitely
    generated",
  journal="Order",
  volume="3",
  year="1986",
  pages="211-218"
}
@article{freese-valeriote-complexity,
  author="Matthew A. Valeriote and Ralph Freese",
  title="On the complexity of some {M}altsev conditions",
  journal="International Journal of Algebra and Computation",
  year="2009",
  volume="19",
  number="2",
  pages="451-464",
  doi="10.1142/S0218196709004956",
  issn="1793-6500"
}
@article{nebojsa,
  author="Nebojsa Mudrinski",
  title="2-Supernilpotent {M}al'cev algebras",
  journal="Monatshefte Fur  Mathematik",
  year="2013",
    issue="2",
    volume="172",
    pages="161--166",
   note="doi:10.1007/s00605-013-0541-y"
}
@INPROCEEDINGS{bulatov-on-the-number-of-maltsevs,
  author="Andrei Bulatov",
  title="On the number of finite {M}al'tsev algebras",
  booktitle="Proceedings of the Dresden Conference 2000 (AAA 60) and the Summer School
    1999 (Contr. Gen. Alg.)",
  pages="41–54",
  publisher="Verlag Johannes Heyn", 
  year="2001",
  url="https://www.cs.sfu.ca/~abulatov/papers/CGA13.ps"
}
@article{kearnes-szendrei-2-commutators,
  author="Keith Kearnes and Agnes Szendrei",
  title="The relationship between two commutators",
  journal="International Journal of Algebra and Computation",
  year="1998",
  volume="8",
  pages="497--531"
}
@article{hagemann-mitschke,
  author="J. Hagemann and A.  Mitschke",
  title="On $n$-permutable congruences",
  journal="Algebra Universalis",
  volume="3",
  number="1",
  pages="8--12",
  year="1973"
}
@article{freese-mckenzie-robust,
  author="Ralph Freese and Ralph McKenzie",
  title="Maltsev families of varieties closed under
   join or {M}altsev product",
  note="preprint",
 url="http://www.math.hawaii.edu/~ralph/Preprints/maltsev-robust.pdf"
}
@article{larose-zadori,
  author="Benoit Larose and La\'szlo Z\'adori",
  title="Algebraic properties and dismantlability of finite posets",
  journal="Discrete Mathematics",
  year="1997",
  volume="163",
  pages="89--99",
  note="doi:10.1016/0012-365X(95)00312-K"
}
\end{filecontents*}

\documentclass[12pt]{article} 
\usepackage{hyperref}
  \usepackage{tikz}
  \usepackage[english]{babel}
  \usepackage[utf8]{inputenc}
  \usepackage{amsfonts,amsmath,amssymb,amsthm,enumerate,latexsym}

  \let\phi\varphi
  \def\zet{{\mathbb Z}}
  \def\Im{\operatorname{Im}}
  \def\equals{=}
  \DeclareMathOperator\absorbs{\trianglelefteq}
  \DeclareMathOperator\Gabsorbs{\trianglelefteq_G}
  \DeclareMathOperator\Jabsorbs{\trianglelefteq_J}
  \let\subset\subseteq
  \def\compEXPTIME{{\textsf{EXPTIME}}}
  \def\compNEXPTIME{{\textsf{NEXPTIME}}}
  \def\compNP{{\textsf{NP}}}
  \def\compP{{\textsf{P}}}
  \def\Datalog{{\textsf{DATALOG}}}
  \let\nicepicture\relax
  \def\Operations{{\mathcal F}}
  \def\algA{{\mathbf{A}}}
  \def\algM{{\mathbf{M}}}
  \def\algB{{\mathbf{B}}}
  \def\algC{{\mathbf{C}}}
  \def\A{{\mathbb{A}}}
  \def\B{{\mathbb{B}}}
  \newcommand\C{{\mathcal C}}
  \newcommand\vect[1]{\mathbf {#1}}
  \newcommand\Arrr{{\mathcal R}}
  \newcommand\M{{\mathcal M}}
  \newcommand\AND{\mathrel{\wedge}}
  \newcommand\OR{\mathrel{\vee}}
  \newcommand\True{\mathrm{True}}
  \newcommand\en{\mathbb{N}}
  \DeclareMathOperator\CSP{CSP}
  \DeclareMathOperator\Sg{Sg}
  \DeclareMathOperator\arity{arity}
  \DeclareMathOperator\Inv{Inv}
  \DeclareMathOperator\Pol{Pol}
  \DeclareMathOperator\var{var}
  \DeclareMathOperator\NU{NU}
  \DeclareMathOperator\Con{Con}
  \DeclareMathOperator\Clo{Clo}
  \newcommand\SDmeet{$\operatorname{SD}(\wedge)$}

  %===math symbols
  \def\Rplus{R^{+}}
  \def\Rminus{R^{-}}
  \def\M{{\mathcal M}}
  \def\CSP{\operatorname{CSP}}
  \def\sinks{\operatorname{sinks}}
  \def\sources{\operatorname{sources}}
  \def\var{\operatorname{var}}


  \theoremstyle{definition}

  \newtheorem{question}{Question}

  \theoremstyle{remark}
  \newtheorem*{context}{Context}


\title{Vanderbilt Open Problems List 2015}
\author{}
\begin{document}
\maketitle
%===============================================
\section{Maltsev conditions}
%===============================================


\begin{question}[Matt Valeriote]
  Is there a strong Maltsev condition for a difference term for finite
algebras? If so, what is it?
\end{question}
\begin{context}
  It is known~\cite{kearnes-szendrei-2-commutators} that having a difference term corresponds to a Maltsev condition.
\end{context}

\begin{question}[Ralph McKenzie]
Let $V$ be a variety. Is it true that $V$ has a difference term if and only if
for every $\algA\in V$ and ever every $\alpha,\beta\in \Con (\algA)$ with 
$\alpha$ Abelian we have $\alpha\circ\beta=\beta\circ\alpha$?
\end{question}
\begin{question}[Ralph McKenzie]
  Is there a $M$ strong Maltsev condition such that a variety $V$ has a Taylor
term iff $V$ satisfies $M$?
\end{question}

\begin{question}[Ralph McKenzie]
  Is there a $M$ strong Maltsev condition such that a variety $V$ is congruence
  meet semidistributive (\SDmeet) iff $V$ satisfies $M$?
\end{question}
\begin{context}
Both above questions have positive answers for locally finite varieties.
\end{context}
\begin{question}[Libor Barto]
  Is congruence singularity (CS) characterized by a linear Maltsev condition
for finitely generated varieties?
\end{question}
\begin{context}
  CS means that for all (finite) $\algA\in V$ and all $\alpha,\beta\in \Con (\algA)$, we
have
\[
\left|a/\alpha\wedge \beta \right|\cdot\left|a/\alpha\vee\beta \right|=
 \left|a/\alpha\right|\cdot\left|a/\beta \right|.
\]
 
  CS is known to be a Maltsev condition for finitely generated varieties.
\end{context}

\begin{question}[Libor Barto]
  Fill in the blanks: `` \dots $+$ finitely related = \dots''.
\end{question}
\begin{context}
Two famous ways to fill in the blanks are 
``CM $+$ finitely related = few subpowers'' and ``CD $+$ finitely related =
NU''. Are there more?
\end{context}
\begin{question}[Libor Barto]
Let $\algA$ be an algebra that is
\begin{enumerate}[a)]
\item idempotent and topologically closed (i.e.. if $f\colon A^n\to A$ is an
operation such that for any finite $B\subset A$ we have a $g$ operation of
$\algA$ such that $g_{|B}=f_{|B}$ then $f$ is also an operation of $\algA$), or
\item oligomorphic (i.e. ) and topologically closed.
\end{enumerate}
If the 2-element naked set (no nontrivial
idempotent operations) can be found in $HSP(\algA)$, can we find it in
$HSP_{fin}(A)$? 
\end{question}

\begin{context}
This double question is interesting for infinite algebras (it is known to be true for
finite algebras). If having a Taylor term is not a strong Maltsev condition
(which we don't know) then the answer to a) above is ``no''.

Naked set is an algebra with no nontrivial (non-projection) idempotent
operations.

An algebra is oligomorphic if the following holds: Let $G$ be the group of all
unary terms of $\algA$ for which there is an inverse term in $\algA$. Then for
all $n\in\en$ the number of orbits of $G$ acting componentwise on $A^n$ is
finite. One can't have an infinite algebra that would be
oligomorphic and idempotent at the same time.

In general, there are algebras that have the the 2-element naked set in
 $HSP(\algA)$, but not in $HSP_{fin}(\algA)$.
\end{context}
\begin{question}[Alex Kazda]
  Let $\algA$ be a finite 4-permutable algebra. Find an algorithm that solves
$CSP(\algA)$ in polynomial time.
\end{question}
\begin{question}[Jakub Opršal]
 Give a nice examples of 4-permutable finite algebras that are not congruence meet
semidistributive, nor congruence modular.
\end{question}
\begin{question}[Alex Wires]
  Let $\algA$ be a 3-permutable algebra. Find an algorithm for $\CSP(\algA)$
that does not need to use the edge term.
\end{question}
\begin{question}[Ross Willard]
  Does there exist a polynomial time algorithm that, given a $G$ finite
bipartite graph would decide if $G$ has a set of Hagemann-Mitschke
polymorphisms?
\end{question}
\begin{context}
  Hagemann-Mitschke terms witness $n$-permutability ~\cite{hagemann-mitschke}.

  The goal here is to generalize from bipartite graphs to general graphs and
to do the same for congruence modularity.
\end{context}

\begin{question}[Jakub Opršal]
  Is congruence modularity a linear meet of Maltsev and congruence
distributivity?
\end{question}
\begin{context}
The meet here is taken in the lattice of interpretability of linear Maltsev
conditions. Meet of two Maltsev conditions is the strongest linear Maltsev
condition implied by each of the two original conditions.
\end{context}

\begin{question}[Ross Willard]
  Why are all interesting Maltsev conditions expressible in two variable
equations?
\end{question}
\begin{question}[Libor Barto]
  Which implications among Maltsev conditions can be extended from the locally
finite to oligomorphic closed clones by way of oligomorphic adjustment by
automorphisms (i.e. $t(x,y)=t(y,x)$ becomes $t(x,y)=\sigma
t(\theta(y),\lambda(x))$)? Here $\sigma, \theta, \lambda$ belong to the group of all
unary terms of $\algA$ for which there is an inverse term in $\algA$
\end{question}
\begin{context}
  We prefer closed clones, because they precisely correspond to polymorphisms
of relational structures.
\end{context}
\begin{question}[Matt Moore]
  Which finite \SDmeet{} algebras are finitely related?
\end{question}
\begin{context}
  We know that having a cube term is a sufficient condition here, but it is not
necessary (consider the two element semilattice).
\end{context}
\begin{question}[Ralph Freese]
 Let $V,W$ be idempotent congruence permutable varieties. Find 
Hagemann-Mitschke terms and the cube term for the variety $V\circ W$ 
(where $\circ$ denotes Maltsev product of varieties).
\end{question}
\begin{context}
  It is known (\cite{freese-mckenzie-robust}) that if $V,W$ are idempotent congruence permutable
  varieties then $V\circ W$ (join of varieties) is 3-permutable (it need not be permutable), but
nobody knows how to write the Hagemann-Mitschke term for $V\circ W$ as a
composition of Maltsev operations in $V,W$. 

We also know that $V\circ W$ has a cube term since having a cube term (we don't
need 3-permutability to prove this here since cube terms are robust with
respect to $\circ$).
\end{context}

\begin{question}[Cliff Bergman]
  Assume that $V,W$ are idempotent varieties and that $V\vee W$ (join of
  varieties of the same type with respect to inclusion; \emph{not} with respect to
  interpretability) is 3-permutable. Is it then true that $V\circ W$ is 3-permutable?
\end{question}
\begin{context}
  It is known that for $V,W$ idempotent we have ``$V\vee W$ permutable
$\Rightarrow$ $V\circ W$ permutable.''
\end{context}

\begin{question}[folklore]
  Let $P$ be a finite bounded poset, let $\cal C$ be the clone of polymorphisms
of $P$. Is it true that $\cal C$ is a finitely generated clone iff $\cal C$
contains an NU?
\end{question}
\begin{context}
``$\Leftarrow$'' is known by the Baker-Pixley theorem.

Smallest poset without NU is the Tard\H{o}s' poset~\cite{tardos}. The clone of polymorphisms
of that poset does not have NU and is not finitely generated.

Also known is Larose and Zádori's characterization of posets admitting an
NU~\cite{larose-zadori}.
\end{context}

\begin{question}[Matt Valeriote]\label{qLocal}
  We say that a strong Maltsev condition $M$ is local for idempotent algebras
  if there exists an $n$ such that for every idempotent $\algA$ if for every
  $B\subset A$ of size at most $n$ there exist terms of $\algA$ that satisfy
  $M$ for elements of $B$ then $\algA$ as a whole satisfies $M$.

  Which strong Maltsev conditions are local for idempotent algebras?
\end{question}
\begin{context}
  The only provably non-local strong Maltsev condition we have so far is
  minority operation (see Question~\ref{qMinority}). On the other hand, we know
  that Maltsev operation, $n$-permutability (Hagemann-Mitschke terms), $NU(k)$, cube term, 
  $CD(k)$ and $CM(k)$ are local for
  idempotent algebras. (Weak near unanimity of fixed arity is local if we
  restrict ourselves to Taylor idempotent algebras.)
\end{context}

\begin{question}[Alex Kazda]\label{qMinority}
  What is the computational complexity of deciding if $\algA$ finite idempotent
  algebra given by tables of its basic operations has a minority operation?
\end{question}
\begin{context}
  We know that having local minorities (see Question~\ref{qLocal}) 
  is not enough thanks to Dmitriy Zhuk's example of idempotent 
  algebras that have local minority operations (there is such an algebra for
  each value of $n$) but no global minority.
\end{context}

%============================================
\section{Clones}
%============================================
\begin{question}[Ross Willard] Ralph McKenzie provided
  in~\cite{mckenzie-residual-bounds} an algebra
  $\algA$ that has 4 elements, the variety generated by $\algA$ has residual
  character $\omega$ and $\Clo \algA$ is not finitely generated. Decide if
  $\Clo \algA$ is finitely related. (Guess: No.)
\end{question}

\begin{question}[Dmitriy Zhuk] Fix a base set $A$. Let $\mathcal E$ 
  be the set of all clones on $A$ generated by a single edge operation. Let $C$
  be an inclusion minimal member of $\mathcal E$. Is it true that $C$ is always
  generated by a 4-ary edge operation?
\end{question}
\begin{question}[Dmitriy Zhuk] Fix a base set $A$. Let $\mathcal E$ 
  be the set of all clones on $A$ generated by a single NU operation. Let $C$
  be an inclusion minimal member of $\mathcal E$. Is it true that $C$ is always
  generated by a majority operation?
\end{question}

\begin{context}
  An algebra $\algA$ has a $k$-ary NU if and only if $\algA$ is CD and has
  $k+1$-ary edge term, which makes the switch from 4-ary edge term to a
  majority operation plausible.

  Both above questions have a positive answer when $|A|=2$. In the Post's
  lattice, the only clone that has an edge term and does not have a subclone
  with an edge term is the clone generated by $p(x,y,z)=x+y+z\pmod 2$ and there
  is no clone that would be minimal among all clones that have an edge term.
\end{context}

\begin{question}[Dmitriy Zhuk]
  Give complexity classification or practical algorithms that decide if two definitions give the same clone. Cases:
  \begin{itemize}
    \item Given two finite sets of operations $F_1, F_2$, decide if
      $\Clo(F_1)=\Clo(F_2)$. (This is decidable.)
    \item Given two finite relations $\rho_1, \rho_2$, decide if
      $\Pol(\rho_1)=\Pol(\rho_2)$. (This is decidable.)
    \item Given a finite relation $\rho$ and a finite set of operations $F$,
      decide if $\Pol(\rho)=\Clo(F)$. (We don't know if this is even
      decidable.)
    \item Given three finite sets of operations $A, B, C$, decide if 
      $\Clo(A)\cap \Clo(B)=\Clo(C)$. (We don't know if this is even
      decidable.)
  \end{itemize}
\end{question}

\begin{question}[Dmitriy Zhuk]
  Given a clone, decide how many clones contain this clone. Variants:
\begin{itemize}
    \item Given a finite set of operations $F$, decide how many clones contain
      $\Clo(F)$. (Decidability unknown.)
    \item Given an operation, decide how many clones contain this operation.
      (Decidability unknown.)
    \item Given a weak near unanimity operation, decide how many clones contain this operation.
      (Decidability unknown.)
    \item Given a Maltsev operation, decide how many clones contain this operation.
      (Decidability unknown.)
    \item Given a relation $\rho$, decide how many clones contain $\Pol(\rho)$.
      (Decidability unknown.)
  \end{itemize}
\end{question}

\begin{question}[Dmitriy Zhuk]
  Find out how many clones contain a given minimal clone on 4 elements.
\end{question}
\begin{context}
All minimal clones on 4 elements were found (by combining results of 
P\"oschel, Kaluzhnin, Szczepara, Waldhauser, Jezek, Quackenbush and Karsten
Scholzer, the last from 2012)
\end{context}
\begin{question}[Dmitriy Zhuk]
  Find all minimal clones that are contained only in \emph{finitely} many clones (i.e.
  correspond to a finite principal filter in the clone lattice).
\end{question}

\begin{question}[Dmitriy Zhuk]
  Find all minimal clones that are contained only in \emph{countably} many clones (i.e.
  correspond to a finite principal filter in the clone lattice).
\end{question}

\begin{question}[Dmitriy Zhuk]
  Decide if a clone is finitely generated. Cases:
  \begin{itemize}
    \item Given a relation $\rho$, decide if $\Pol(\rho)$ is finitely
      generated. (Decidability unknown.)
    \item As above, but assume that $\rho$ is a partial order relation with
      greatest and least element. (Decidability unknown.)
    \item Given two sets of operations $B, C$, decide if $\Clo(B)\cap \Clo(C)$
      is finitely generated. (Decidability unknown.)
  \end{itemize}
\end{question}
\begin{context}
  See~\cite{haddad} for finitely generated clones whose intersection is not
  finitely generated.
\end{context}

\begin{question}[Dmitriy Zhuk]
Given a clone by a set of operations, decide whether the clone is finitely related. (Decidability unknown.) 
\end{question}

\begin{question}[Dmitriy Zhuk]
Given a clone $C$ by a finite list of operations and knowing that $C$ is
finitely related, find a relation that defines this clone. (Complexity unknown.)
\end{question}
\begin{question}[Dmitriy Zhuk]
  Given a relation $\rho$ such that $\Pol(\rho)$ is finitely generated, find
  a generating set for $\Pol(\rho)$. (Complexity is unknown.)
\end{question}

\begin{question}[Dmitriy Zhuk]
  Does there exist a minimal clone that is not finitely related? Does 
  there exist a minimal clone on 3 elements that is not finitely
  related?
\end{question}

\begin{question}[Dmitriy Zhuk]
  Given two finite sets of operations $C$ and $D$ such
  that $C\subset D$ and the sublattice between two clones
  $\Clo(C)$ and $\Clo(D)$ is finite, find this sublattice. (Decidability
  unknown.)
\end{question}

\begin{question}[Dmitriy Zhuk]
  Does there exist a finite maximal chain of size greater than 52 on 3
  elements?
\end{question}
\begin{question}[Dmitriy Zhuk]
  Is it true that a finite maximal chain of the maximal size always
  contains a minimal majority operation?
\end{question}
\begin{context}
  The two above questions stem from the fact that there is a 52 element maximal
  chain of clones on three elements with a minimal majority operation.
\end{context}

\begin{question}[Dmitriy Zhuk]
  Is the size of a finite maximal chain of clones on $A$ finite bounded?
\end{question}
\begin{context}
  The bound is 7 when $|A|=2$, we don't know for $|A|=3$.
\end{context}

\begin{question}[Dmitriy Zhuk]
  Find a set of relations $\Gamma$ of arity at most $m$ such that
  $\Pol(\Gamma)$ is finitely generated, but any generating set
  of $\Pol(\Gamma)$ has to contain an operation of arity greater than $|A|^m$ (or,
  alternatively, $2^{2^m}$).
\end{question}

\begin{question}[Dmitriy Zhuk]
  Find a set of operations $C$ of arity at most $m$ such that $\Clo(C)$ is
  finitely related, but whenever
  $\Clo(C)=\Pol(\Gamma)$ for some $\Gamma$ set of relations, then $\Gamma$
  contains a relation of arity greater than $|A|^m$ (or, alternatively,
  $2^{2^m}$).
\end{question}


%============================================
\section{Commutators}
%============================================
\begin{question}[Alex Wires]
Generalize results from commutator theory to the $n$-ary commutator by
Bulatov~\cite{bulatov-on-the-number-of-maltsevs} (see Question~\ref{qSupernilpotent} for a concrete example of
that).
\end{question}

\begin{question}[Alex Wires]\label{qSupernilpotent}
  Fill in the blank: An algebra $\algA$ is supernilpotent if and only if
  $\algA$ is polynomially equivalent to \dots
\end{question}
\begin{context}
  One guess for the ``$\dots$'' above is ``affine algebra''. See the results on
  2-supernilpotent algebras by Nebojsa Mudrinski~\cite{nebojsa}.
\end{context}


%============================================
\section{Algorithmic topics}
%============================================
\begin{question}[Ross Willard]
  Is the following decidable? Given $\algA$ finite algebra and $n\in \en$, are
all subdirectly irreducible algebras in the variety generated by $\algA$ of
cardinality at most $n$?
\end{question}
\begin{context}
  The problem is decidable in the \SDmeet{} and CM cases (because then we can
decide if $\algA$ has a finite residual bound).
\end{context}
\begin{question}[Matt Valeriote]
  Find a strong idempotent nontrivial Maltsev condition $M$ such that deciding
  if an algebra $\algA$ satisfies $M$ is provably \emph{not}
  \compEXPTIME-complete. (Modulo reasonable computational complexity conjectures.)
\end{question}
\begin{context}
  Maltsev (not always strong) conditions known to be \compEXPTIME-complete~\cite{freese-valeriote-complexity}:
  \begin{itemize}
    \item omitting type 1
    \item omitting types 1 and 2
    \item omitting types 1 and 5
    \item omitting types 1, 2, and 5
    \item having a semilattice operation
    \item congruence modularity
    \item congruence distributivity
    \item having a chain of Jónsson terms of fixed length $n$
  \end{itemize}

  Deciding a strong Maltsev condition is in \compEXPTIME, because we can always
  build up the $n$-generated free algebra (with $n$ corresponding to the
  arities of operations in the Maltsev condition) in $V(\algA)$ and examine all its members.
\end{context}
\begin{question}[Matt Valeriote]
  Given a general (not idempotent) algebra $\algA$, how hard is it to decide if
  $\algA$:
  \begin{enumerate}[a)]
    \item has Maltsev operation,
    \item has a majority operation,
    \item has a Pixley operation,
    \item is primal.
  \end{enumerate}
\end{question}
\begin{question}[Matt Valeriote]
  Find a strong idempotent nontrivial Maltsev condition $M$ such that deciding
  if an \emph{idempotent} algebra $\algA$ satisfies $M$ is provably \emph{not}
  in \compP{} (modulo reasonable computational complexity conjectures.), or show
  that we can decide all such Maltsev conditions in polynomial time. 
\end{question}
\begin{context}
  Maltsev (not always strong) conditions known to be in \compP{} for $\algA$
  idempotent:
  \begin{itemize}
    \item  Jónsson terms (directed, undirected, with a fixed length of chain
  or with a chain of any length)
     \item Gumm terms (all variants)
     \item Maltsev operation
     \item $NU(k)$ for a fixed $k$ or any $k$
     \item cube term of fixed dimension or any
  dimension
      \item Pixley term
      \item Hagemann-Mitschke terms for fixed length of chain or any length of
	chain
      \item having some weak near
	unanimity operation (this is due to Bulatov,
	see~\cite[Theorem 6.3]{freese-valeriote-complexity} for an algorithm) 
  and having a weak near unanimity of a fixed arity
      \item having a flat semilattice operation (only two levels)
  \end{itemize}

  Complexity is open for (among others):
  \begin{itemize}
    \item semilattice operation (not linear, conjectured to be
      \compEXPTIME-complete even for $\algA$ idempotent;
      see~\cite{} for an
      algebra with local semilattice operations, but no global semilattice)
    \item minority (this is Question~\ref{qMinority})
    \item Having totally symmetric idempotent operation of arity $k$ for $k$
      fixed
    \item omitting types 1 and 5
    \item Day terms  (the last two should be easy do decide).
  \end{itemize}
\end{context}
\begin{question}
  What is the complexity of deciding if a given relational structure has a
  Maltsev polymorphisms?
\end{question}
\begin{question}
  How hard is it to tell if a relational structure has a Taylor polymorphism?
\end{question}
\begin{question}
  How hard is it to tell if a relational structure has a semilattice
  polymorphism?
\end{question}
%========================================
\section{McKenzie-Wagner Circuits}
%========================================
In this section, $\algA^{+}$ is the algebra with the universe $2^\omega$ and basic
operations $\cap$, $\cup$ (binary set operations), $+$, $\cdot$ (binary
elementwise 
arithmetics; for
example $\{1,2\}+\{4,5\}=\{5,6,7\}$), ${}^{c}$ (complement) and constants 
$\emptyset$, $\omega$, $\{0\}$, and $\{1\}$.

The algebra $\algM$ is the minimum subalgebra of $\algA^{+}$, i.e. $\algM$ contains
all sets we can describe using $\{\emptyset,\omega, \{0\},\{1\}\}$ and basic
operations.

$\algM$ has many interesting members: finite sets, cofinite sets, all primes,
\dots
\begin{question}[Cliff Bergman]
  Does $\algM$ contain $\{u^2\colon u\in\omega\}$?
\end{question}
\begin{question}[Cliff Bergman]
  Is the word problem in $\algM$ decidable? Or, to put it differently, can we
  decide if a term of $\algM$ describes the empty set (problem EMPTY)?
\end{question}
\begin{context}
  (Pierre) McKenzie and Wagner have shown that EMPTY is \compNEXPTIME-hard, and
  \compNEXPTIME-complete when we leave out the complement operation.

  Among instances of EMPTY is the Goldbach conjecture.
\end{context}
\bibliographystyle{plain}
\bibliography{citations}

\end{document}
